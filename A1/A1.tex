\documentclass{article}
\usepackage[utf8]{inputenc}
\usepackage{amsmath}
\usepackage{mathtools}
\usepackage{bm}
\title{COL352: Assignment 1}
\author{Sachin 2019CS10722 }
\date{January, 2022}

\begin{document}

\maketitle

\section{Question 1}

\textbf{Given an alphabet $\Gamma $ = $\{l_1,...,l_k\}$, construct an NFA that accepts strings that don't have
all the characters from $\Gamma $. Can you give an NFA with k states?}\\

Consider the NFA N = {$(Q, \Sigma, \delta, q_0 , F )$} where,\\
$Q = 2^\Gamma = \{\phi, \{l_1\}, \{l_2\}, ....... \Gamma\}$\\
$\Sigma = \Gamma$\\
$q_0 = \phi$\\
$F = Q \setminus  \Gamma$\\
$\delta \subseteq Q \times \Sigma \times Q$ defined as follows:\\

\begin{equation}
    \delta (q,a) = 
    \begin{cases*}
      q & if $a \in q$ \\
      q \bigcup \{a\}        & otherwise
    \end{cases*}
\end{equation}

\textbf{Claim: } N accepts only those strings that don't have all the characters from $\Gamma $.\\

\textbf{Proof: } \\

    Let S = $x_1 x_2 ..... x_k$ be any arbitrary string, let's consider the run of s on N.\\
    $\phi \xrightarrow{x_1} \{x_1\} \xrightarrow{x_1} \{x_1,x_2\} ....... \xrightarrow{x_k} \bigcup \{x_i\}$\\
    NOTE: \{\} is a set in above expression, and $\bigcup \{x_i\}$ will contain single copy of each alphabet.\\
    \begin{enumerate}
        \item Case 1: $\bigcup \{x_i\} = \Gamma$ \\
        This means that string s contains all the alphabets from $\Gamma $.\\
        Also, final state of N = $\bigcup \{x_i\} = \Gamma \notin $ F.\\
        Hence, N does not accepts the string s.
        \pagebreak
        \item Case 2: $\bigcup \{x_i\} \subset \Gamma$ \\
        This means that string s does not contains all the alphabets from $\Gamma $.\\
        Also, final state of N (let $p_k$) = $\bigcup \{x_i\} \neq \Gamma => p_k \in $ F.\\
        Hence, N accepts the string s.\\
    \end{enumerate}

Hence proved that N accepts only those strings that don't have all the characters from $\Gamma $.\\
\pagebreak
% -------------------------------------------------
\section{Question 2}
\textbf{An all-NFA M is a 5-tuple \boldsymbol{$(Q, \Sigma, \delta, q_0 , F )$} that accepts \boldsymbol{$x \in \Sigma^{*}$} 
if every possible state that M could be in after reading input x is a state from F . Note, in contrast, that an ordinary NFA 
accepts a string if some state among these possible states is an accept state. Prove that all-NFAs recognize the class of regular languages.}\\
\\
To prove that all-NFA's recognise the class of regular languages we need to show two things, firstly that the language accepted 
by all-NFA's is regular, and secondly given any regular language there exists an all-NFA which accepts it. Following are the proofs of these parts,
\\
\\
\textbf{To Prove:} Language accepted by all-NFA is regular.\\
\textbf{Proof:} Now by the definition, all-NFA M is a 5-tuple $(Q, \Sigma, \delta, q_0 , F )$ that accepts $x \in \Sigma^{*}$ if 
every possible state that M could be in after reading input x is a state from F . This would mean the all-NFA's are NFA because NFA a
ccepts the string even if some of the states reached after reading an input x is in accept state F. NOw we know that the language 
accepted by NFA is regular. Therefore the language accepted by all-NFA is also regular. Hence proved.\\
\\
\textbf{To Prove:} For every regular language there exists an all-NFA that accepts it.\\
\textbf{Proof:} We know that for every regular language there exists a DFA which accepts it. Now the definition of a 
DFA M is that it is a 5-tuple $(Q, \Sigma, \delta, q_0 , F )$ that accepts $x \in \Sigma^{*}$ if the state that M could 
be in after reading input x is a state from F. Now we also know that the set of states DFA M would be in after reading the 
input x is a singleton set (Deterministic nature) and the state belongs to F if x is accepted by DFA. So every DFA is an
 all-NFA. Therefore for every regular language, there exists an all-NFA that accepts it. Hence proved.\\
\\
Now above two facts would imply that the all-NFA's recognize the class of regular languages.


\pagebreak
% ------------------------------------------------
\section{Question 3}





\pagebreak
% ------------------------------------------------
\section{Question 4}




\pagebreak
% ------------------------------------------------
\section{Question 5}

\textbf{For any string \boldsymbol{$w = w_1 w_2 . . . w_n$} the reverse of w written \boldsymbol{$w^R$} is the string 
\boldsymbol{$w_n . . . w_2 w_1$}. For any language A, let \boldsymbol{$A^R ={w^R | w \in A}$}. Show that if A is regular,
 then so is \boldsymbol{$A^R$}. In other words, regular languages are closed under the reverse operation.}\\
\\
As every regular language has a DFA which accepts it, let D be $(Q, \Sigma, \delta, q_0, F )$ that accepts the language
 A. Now we will construct an NFA N from this DFA. The steps of the construction are given below. We will then show that 
 the language accepted by NFA is indeed $A^R$.\\
\\
\textbf{Construction:} To construct this NFA N we will have to reverse all the edges of the DFA D. Also make the start 
state of the D as the accepting state of the N. Add $\epsilon$ transitions from the accepting states of D to a new state 
$s$ in NFA. Make $s$ the start state of the NFA.\\
So N is $(Q_1, \Sigma, \delta^{'}, q_0^{'} , F^{'} )$ such that
\begin{equation}
\begin{split}
&            Q_1=Q\cup\{s\}\\
&           F^{'}=q_0\\
&           \delta^{'}(q_1,a) = 
            \begin{cases*}
            q_2 & if $q_1 \in Q \setminus F$  and  $\delta(q_2,a)=q_1$\\
            s       & if $ q_1 \in F$ and $a=\epsilon$
            \end{cases*}\\
&           q_0^{'}=s \\
\end{split}
\end{equation}
\textbf{To Prove:} The language accepted by N is $A^R$.\\
\textbf{Proof:} Now take any string $w_1 w_2....w_n$ from the language A. The path(path is state then alphabet taken then
 next state reached) taken by this string to accept state in D would be $q_0,w_1,q_1,w_2,q_2.....q_n,w_n,f$ where $q_i\in
  Q$ and $f\in F$. Now take the reverse of the string $w_n w_{n-1}....w_1$. There exist a path from start to accept state 
  in N as which is as follows $s,\epsilon,f,w_n,q_n,w_{n-1}.....q_1,w_1,q_0$. We know that $q_0$ is an accept state of N.
  Thus we have got an NFA that has an accepting path for any string w in $A^R$.\\
Now we have proved that for every string in A we have a accepting path in N. Now we can also similarly prove the reverse 
direction too i.e. for every string w in $A^R$ we have accepting path in D for reverse of w. The proof of this goes as 
follows:\\ Take any string $w_1 w_2....w_n$ from the language $A^R$. The pathtaken by this string to accept state in N 
would be $s,\epsilon,f,w_1,q_n,w_2.....q_1,w_n,q_)$ where $q_i\in Q$ and $f\in F$. Now take the reverse of the string $w_n 
w_{n-1}....w_1$. There exist a path from start to accept state in D as which is as follows $q_0,w_n,q_1,w_{n-1},q_2.....q_n,w_1,
f$. We know that f is an accept state of D.Thus we have got an DFA that has an accepting path for any string w in A.\\
Hence we have proved that regular languages are closed under the reverse operation. 


\pagebreak
% ------------------------------------------------
\section{Question 6}

\textbf{Let \boldsymbol{$\Sigma$} and \boldsymbol{$\Gamma$} be two finite alphabets. A function \boldsymbol{$f : \Sigma^{*} \to \Gamma^{*}$ }is called a homomorphism if
for all x,y \boldsymbol{$ \in \Sigma^{*}, f(x . y) = f(x) . f(y)$}. Observe that if f is a string homomorphism, then
\boldsymbol{$f(\epsilon) =\epsilon$}, and the values of $f(a)$ for all a \boldsymbol{$\in \Sigma$} completely determines \boldsymbol{$f$}. Prove that the class of
regular languages is closed under homomorphisms.That is, prove that if L \boldsymbol{$\subseteq \Sigma^{*} $} is a regular
language, then \boldsymbol{$f(L) = \{f(x) \in \Sigma^{*} | x \in L \}$} is regular. Try to informally describe how you
will start with a DFA for L and get an NFA for f(L).
}

\begin{enumerate}
    \item \boldsymbol{$f(\epsilon) =\epsilon$}
    
    $\forall x,y \in \Sigma^{*} f(x.y) = f(x).f(y)$ (by defination of homomorphism)\\
    take y = $\epsilon$\\
    $=> \forall x \in \Sigma^{*} f(x.\epsilon) = f(x).f(\epsilon)$\\
    $=> \forall x \in \Sigma^{*} f(x) = f(x).f(\epsilon)$\\
    but, any alphabet that upon concatenation with other alphabet generate same alphabet is $\epsilon$ only\\
    $=> f(\epsilon) =\epsilon$\\

    This, is a property of homomorphism that it maps identity elements (of the operation upon which homomorphism property is satisfyied). Here identity element with respect to an operation (say oper) is any element of the set that maps
    every element to itself upon oper.
    In our case, since $a.\epsilon = a \forall a \in \Sigma^{*} $ as well as $a.\epsilon = a \forall a \in \Gamma^{*} $, Hence $\epsilon$ is the identity element of both $\Sigma$ and $\Gamma$

    \item \textbf{values of $f(a)$ for all a \boldsymbol{$\in \Sigma$} completely determines \boldsymbol{$f$}}\\
    
    \textbf{Given: } $f(a) \text{ } \forall \text{ }  a \in \Sigma$\\
    \textbf{To find: } $f(x) \text{ } \forall \text{ }  x \in \Sigma^{*}$\\

    Let x $\in \Sigma^{*}$\\
    $=> x = a_1 a_2 ..... a_k$ where $a_i \in \Sigma $ and $k \geq 0$\\
    $=> f(x) = f(a_1 a_2 ..... a_k)$\\
    $=> f(x) = f(a_1)f(a_2)....f(a_k)$\\
    since, each $a_i \in \Sigma$ hence, they are known.\\
    $=> f(a_1)f(a_2)....f(a_k)$ is known.\\
    $=> f(x) $ is known.\\
    Hence, we can determine $f(x) \forall x \in \Sigma^{*}$

    \pagebreak
    \item \textbf{Prove that the class of regular languages is closed under homomorphisms}

    Let, L be a regular language L, to prove that regular language is closed under homomorphism it is sufficient to
    show that for any homomorphism f, f(L) is also regular.\\
    \textbf{Given:} A regular language L, A homomorphism f\\
    \textbf{To Prove:} f(L) is also regular\\
    \textbf{Proof:}

    Since, L is a regular language, we can construct a DFA (say D) for L.\\
    So, D = {$(Q, \Sigma, \delta, q_0 , F )$} such that, L(D) = L\\
    First let me define a function $g: \Gamma \to \Sigma$ as follows:\\
    $g(a) = \{ x | f(x) = a \} $\\

    Now, Consider the NFA N = {$(Q, \Gamma, \delta', q_0 , F)$} where, \\
    $\delta' \subseteq Q \times \Gamma \times Q$ defined as follows:\\
    $ \delta' (q,a) = \{ \delta (q,x) | x \in g(a) \} $\\

    Proving that L(N) = f(L) \\
    Let, $x \in f(L) => x = f(a) $ where, $a \in L$\\
    Let, $a = a_1 a_2..... a_k$\\
    $=> x = f(a_1) f(a_2) ..... f(a_k)$ \\
    Since, $a \in L = L(D) => \exists \{q_0, q_1,......,q_k\}$ where $d_i \in Q$ such that\\
    $\delta (q_i,a_{i+1}) = q_{i+1} \forall i = 0,.....,k-1$ and $q_k \in F$\\
    by defination of $\delta'$, $\delta' (q_i,f(a_{i+1})) = q_{i+1} \forall i = 0,.....,k-1$\\
    also, $F' = F => q_k \in F'$\\
    hence, N accepts x.\\
    $=> $ N accepts every string x $\in f(L) $\\
    similarly it can be shown that N rejects every string x' $ \notin f(L) $ (as run of N is parallel to D).\\
    Hence L(N) = f(L).

\end{enumerate}





\end{document}
