\documentclass{article}
\usepackage[utf8]{inputenc}
\usepackage{amsmath}
\usepackage{mathtools}
\usepackage{bm}
\usepackage{graphicx}
\usepackage{float}

\usepackage{rotating}

\title{COL352: Assignment 4}
\author{Sachin 2019CS10722 \\
        Saurabh Verma 2019CS50129\\}
\date{March, 2022}

\begin{document}

\maketitle

\section{Question 1}
\textbf{ Show that every infinite Turing-recognizable language has an infinite decidable subset.}\\


\pagebreak


\section{Question 2}

\textbf{Show that single-tape TMs that cannot write on the portion of the tape containing the input
string recognize only regular languages.\\}

\pagebreak


\section{Question 3}

\textbf{Let C be a language. Prove that C is Turing-recognizable iff a decidable language D exists
such that
    \begin{center}
        $C = \{x | \exists y (<x, y> \in D) \}$\\
    \end{center}
}

\pagebreak


\section{Question 4}

\textbf{Say that a variable A in CFL G is usable if it appears in some derivation of some string
w $\in$ G. Given a CFG G and a variable A, consider the problem of testing whether A is
usable. Formulate this problem as a language and show that it is decidable\\}

\pagebreak


\section{Question 5}

\textbf{Consider the problem of determining whether a Turing machine M on an input w ever at-
tempts to move its head left when its head is on the left-most tape cell. Formulate this
problem as a language and show that it is undecidable\\}

To prove that a language is undecidable it is sufficient to show that if it were decidable then it would imply
that some other language that is already known to be undecidable is also decidable.\\
We will follow the similar proof by contradiction and assume language L to be decidable.\\
First let us provide a formal desciption of L:\\
$L = \{ <M,w> | $ M moves its head left on w when its head is on the left-most tape cell $ \}$\\

Now, assume L to be decidable. Then there exists a turing machine T that decides L.\\
Let us try to construct a turing machine T' from T that accepts $A_{TM}$ (that is what we are doing in every undecidablility proof in class)
\begin{enumerate}
    \item Run M on w and if at any transition it accpets w, move to leftmost tape cell and try to move its head to left.\\
        This is done to make sure that on acceptance there is left movement of the head on leftmost tapecell.\\
    \item If at any transition it is on leftmost cell and try to move left, then move rightward (writing the same character that M wanted to write on this transition) and move 
            left without changing tape. More formally:- \\
            $q_iw_1....w_k \rightarrow q_iw'_1...w_k $\ is changed to $q_iw_1....w_k \rightarrow w'_1q_iw_2...w_k \rightarrow q_1w'_1....w_k$\\
            This is done to make sure that there if no left movement if head is on leftmost tape, if state is not accepting.
\end{enumerate} 

Using above construction we've made sure that T' attempts to move its head left from leftmost tapecell on $<M,w>$ if and only if M accepts w.\\
Now, simply run T' on T to decide whether it attempts to move its head left from leftmost tapecell. That way we can decide whether M accepts w or not but that is 
a contradiction since $A_{TM}$ is undecidable (proved in class)\\
Hence L is undecidable.

\pagebreak


\section{Question 6}

\textbf{Consider the problem of determining whether a Turing machine M on an input w ever at-
tempts to move its head left at any point during its computation on w. Formulate this
problem as a language and show that it is decidable.\\}

The language formed by the above problem is:- \\
$L = \{ <M,w> |$ M moves its head left on w at any point \}.\\

To prove that any language is decidable it is sufficient to show that there exists a Turing Machine that 
decides that language. So we will construct a turing machine T that can decide the above language.\\
But before that let's analyse the 2 possible cases of w on M:-

\begin{enumerate}
    \item \textbf{M never moves its head left on w :}
    This means that M always moves its head right on w. Lets see the run of T on w (Let $|w| = n$).\\
    $q_0w \rightarrow aq_1w_2....w_n \rightarrow_* w'q_i \sqcup \rightarrow w' b q_{i+1} \sqcup \rightarrow_{k-1} w' b_1....b_k q_{i+k} \sqcup   $\\
    Now, since number of states and n are finite, if we take k big enough by pegionhole principle $q_{i+k} = q_i$.\\
    Take k = $|Q|+1$ $=>$ by pegionhole principle some of the states will repeat in k transitions, and since next tape alphabet 
    is always $\sqcap$ (since we are always moving right) same cycle will repeat infinite times.\\
    Hence if M don't moves its head left on w in first $|w| + |Q| + 1$ transitions then it will never move its head left on w.\\

    \item \textbf{M moves its head left on w :}
    From the above case it is clear that M moves its head left on w in first $|w| + |Q| + 1$ transitions.\\
    
\end{enumerate}

Now consider the desciption of T on w.\\
\begin{enumerate}
    \item Run first $|w| + |Q| + 1$ transitions of M on T.
    \item If at any time M moves its head left on w then accept.
    \item Otherwise reject.
\end{enumerate}


Point 1 is possible because we can always simulate a turing machine on another turing machine if its description is given (Universal Turing machine).\\
Correctness of the above description is proved earlier by the claim that f M don't moves its head left on w in first $|w| + |Q| + 1$ transitions then it will never move its head left on w.\\
\pagebreak


\section{Question 7}

\textbf{Let $AM_{BCFG} = \{< G > | G$ is an ambiguous CFG $\}$ \\
    Show that $AM_{BCFG}$ is undecidable via a reduction from PCP\\}

\pagebreak


\section{Question 8}
\textbf{In the Silly Post Correspondence Problem (SPCP), the top string in each pair has the same
length as the bottom string. Show that the SPCP is decidable.\\}

\pagebreak



\end{document}

