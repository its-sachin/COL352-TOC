\documentclass{article}
\usepackage[utf8]{inputenc}
\usepackage{amsmath}
\usepackage{mathtools}
\usepackage{bm}
\usepackage{graphicx}
\title{COL352: Assignment 2}
\author{Sachin 2019CS10722 \\
        Saurabh Verma 2019CS50129\\
        Sriram Verma}
\date{March, 2022}

\begin{document}

\maketitle


\section{Question 1}
\textbf{We say that a context-free grammar G is self-referential if for some non-terminal symbol $X$ we have $X \to^* \alpha X \beta$, where $\alpha, \beta \neq \varepsilon$. Show that a CFG that is not self-referential is regular.}



\pagebreak

\section{Question 2}
\textbf{Prove that the class of context-free languages is closed under intersection with regular languages. That is, prove that if \boldsymbol{$L_1$} is a context-free language and \boldsymbol{$L_2$} is a regular language, then \boldsymbol{$L_1 \cap L_2$}
is a context-free language. Do this by starting with a DFA}\\
\newline
Let us suppose there is a CFL L and a regular langauage R. The pushdown automata that accepts L be P=$(S_1,\Sigma,\Gamma,\delta_1,s_1,F_1)$ and the DFA accepting R be D=$(S_2,\Sigma,\delta_2,s_2,F_2)$. Now we have to show that the language $L\cap R$ is CFL. To show this it is enough to provide a PDA that accepts it. So we will construct such a PDA to prove that  $L\cap R$ is CFL. \\
\textbf{To Prove:} $L\cap R$ is CFL. \\
\textbf{Proof:} We will the above hypothesis by construction. The main idea behind the construction of the PDA is that we will run both the original PDA P for L and the DFA D in parallel on the input string and will only accept when the we reach an accepting state both in P and D. The construction of the PDA is described below:\\
\textbf{Construction:} Let the PDA which accepts $L\cap R$ be M=$(S,\Sigma,\Gamma,\delta,(s_1,s_2),F)$. Here S is $S_1 * S_2$ and F is $F_1 *F_2$. The transition function $\delta$ is described as follow: \\
For all the transitions $((p_1,a,\alpha),(p_2,\beta)) \in \delta_1$  and $(q_1,a,q_2) \in \delta_2$ add the transition $(((p_1,q_1),a,\alpha),((p_2,q_2),\beta))$ in $\delta$.\\
Also, for all the transitions $((p_1,\epsilon,\alpha),(p_2,\beta)) \in \delta_1$  and $\forall q \in S_2$ add the transition $(((p_1,q),a,\alpha),((p_2,q),\beta))$ in $\delta$.\\
Here $p_1,p_2 \in S_1$  $q_1,q_2 \in S_2$  $a \in \Sigma$  $\alpha , \beta \in \Gamma$\\
The accepting condition is that the final state reached after reading the input must belong to F.\\
Now our claim is that PDA M exactly recognises every string that is in $L\cap R$. \\
\textbf{Claim:} The PDA M constructed above exactly recognises strings in $L\cap R$. \\
\textbf{Proof:} We will have to show two things first that every string in $L\cap R$ is accepted by M. Lets prove this. Choose any string w$\in L\cap R$. Then its run on the DFA D would be something like $s_2,q1......q_k$ where $q_k \in F_2$, also w would take the PDA M from start configuration($s_1$) to an accepting configuration($q_{k^{'}}$) in some steps. By the way we have constructed the PDA M the computions of M and D will happen in parallel. So (s1,s2) is the start configuration of the PDA. First state in the tuple denotes the state that would have been in the PDA P and second state denotes the state that would have been in the DFA D after reading input upto some point. So when the PDA gets to wun on w. It would take M from from (s1,s2) to $(q_k,q_{k^{'}})$. Now this is an accepting configuration in M by the way we defined F$(F_1 * F_2)$. So all the string  w$\in L\cap R$ are accepted by M.\\
Also we have to prove that all the strings say w that are accepted by M should also be present in $ L\cap R$. We will accept w if it takes PDA M from $(s_1,s_2)$ to $(q_1,q_2)$ , $q_1 \in F_1$ and $q_2 \in F_2$. Also we have showed that M is parallely running P and D where first state in the tuple means the state reached in P and second state means the state reached in D after reading the input uptill that point. So after reading  w if the M is in state $(q_1,q_2)$, it would mean that after reading w P would have been in $q_1$ and D would be in $q_2$. $q_1 \in F_1$, so $w \in L$ also $q_2 \in F_2$ so $w \in R$ which implies $w \in L \cap R$.\\
\newline
Thus we have successfully constructed a PDA M which accepts $L \cap M$. Thus CFL's are closed under intersection with regular languages. Hence proved. \\
\pagebreak


\section{Question 3}
\textbf{Given two languages $L, L'$, denote by 
$$L||L' := \{x_1y_1x_2y_2 \dots x_ny_n \mid x_1x_2 \dots x_n \in L, y_1y_2\dots y_n \in L'\}$$
Show that if $L$ is a CFL and $L'$ is regular, then $L || L'$ is a CFL by constructing a PDA for $L || L'$. Is $L || L'$ a CFL if both $L$ and $L'$ are CFLs? Justify your answer.}


\pagebreak


\section{Question 4}
\textbf{For \boldsymbol{$A \subseteq \Sigma^*$}, define 
\boldsymbol{$cycle(A) = \{yx \mid xy \in A\}$}
For example if \boldsymbol{$A = \{aaabc\}$}, then 
\boldsymbol{$cycle(A) = \{aaabc, aabca, abcaa, bcaaa, caaab\}$}
Show that if \boldsymbol{$A$} is a CFL then so is \boldsymbol{$cycle(A)$}} \\
\\
Let us suppose that we do have a CFG M=(V,T,P,S) for the langauge A in chomsky normal form. We know for a fact that since A is CFL it will have a CSG. Now to prove that cycle(A) is also a CFL. So we will construct a CFG for cycle(A) that would show that cycle(A) is CFL. \\
\textbf{To Prove:} Cycle(A) is CFL.\\
\textbf{Proof Idea:} Let us consider any string w in language A of the form $x_1 x_2$. Lets look at the parse tree of w. If we turn the parse tree upside down from the leftmost non terminal leaf from where $x_2$ starts then we will get the parse tree for $x_2 x_1$ , which is exactly what we are after. So through the construction described below we try to achieve this affect.\\
\textbf{Construction:} Let us consider the new grammar $M^{'} = (V^{'},T,P^{'},S_0)$ to accept cycle(A). Now here, $ V^{'} = V \cup \{ Z^{'} for \ Z \in V \} \cup \{ S_0\}$\\ P is defined as follow: 
\begin{itemize}
    \item All the rules in P
    \item $S_0 \rightarrow S$
    \item $S^{'}$ $\rightarrow$ $\epsilon$
    \item if  P  contained $ Z \rightarrow a$ add $S_0 \rightarrow aZ^{'}$
    \item if P has $Z \rightarrow XY$ add $Y^{'} \rightarrow Z^{'}X$  and $X^{'} \rightarrow YX^{'}$
\end{itemize}
Now we have constructed the grammar $M^{'}$. Whats left is to show that L($M{'}$) is exactly cycle(A). \\
\textbf{Claim:} L($M^{'}$) is same as cycle(A). \\
\textbf{Proof:} 

\pagebreak

\section{Question 5}
\textbf{Let $$A = \{wtw^R\mid w,t, \in \{0,1\}^* \text{ \ and \ } |w| = |t|\}$$ Show that $A$ is not a CFL. }\\

We will prove that A is not CFL by contrapositive of pumping lemma. i.e. we need to show the following:- \\
$\forall p \geq 0$ \\
$\exists s \in A : |s| \geq p $ \\
$\forall uvxyz = s : |vy| > 0, |vxy| \leq p$\\
$\exists i : uv^ixy^iz \notin A$\\

Let $s = 0^n 1^{\frac{n}{2}} 0^{\frac{n}{2}} 0^n$ ($w = w^R = 0^n, t = 1^{\frac{n}{2}} 0^{\frac{n}{2}}, n = $ any even number more than p)\\
Now, lets divide s in 3 parts = wab, where a = $1^{\frac{n}{2}}$ and b = $0^{\frac{n}{2}} 0^n$\\
Considering all partitions of s = uvxyz such that $|vy| > 0, |vxy| \leq p \leq n$\\
\begin{enumerate}
    \item \textbf{Case 1:} $vxy \subseteq w $ \\
    Considering $s' = uxz$ i.e. i = 0\\
    For s' to be in A, its should be divided into 3 halves = w't'd' of equal length such that first and third are reverse.\\ 
    Since, we have removed characters from w only, w' will have all the characters of w left after removal of v,y and also some 1's 
    from t for it to have same length as other subparts of equal length (see fig).\\
    Hence, $1 \in w'$, but $d' \subset w^R = 0^n => 1 \notin d' => d' \neq w'^R$\\
    Hence, $s' \notin A$\\
    
    \item \textbf{Case 2:} $vxy \subseteq b $ \\
    Considering $s' = uv^2xy^2z$ i.e. i = 2\\
    Let, $|s| = l, |s'| = l', l' < l$\\
    $|w| = \frac{l}{3} < \frac{l'}{3}$\\
    $=> w \subset w' => w' $ will overflow towards t $=> 1 \in w'$\\
    But, $d' \subset w^R $ (after pumping ) that only has 0 (see fig)\\
    $=> d' \neq w'^R$\\
    Hence, $s' \notin A$\\

    \item \textbf{Case 3:} $vxy \bigcap a \neq \phi$\\
    
    \begin{enumerate}
        \item $w \bigcup vxy = \phi$\\
        Considering $s' = uv^2xy^2z$ i.e. i = 2\\
        This case is similar to case 2, as length of string is increased but w remains unchanged, hence w'[1:n] = w = $0^n$ and w'[n+1] = 1.
        But $(n+1)^{th}$ of d' from last = last bit of t = 0.
        $=> d' \neq w'^R$\\
        Hence, $s' \notin A$\\

        \item $w \bigcup vxy \neq \phi$\\
        Considering $s' = uxz$ i.e. i = 0\\
        This case is similar to case 1, as length of string is decreased (all from w). Hence, $1 \in w'$, but $1 \notin d'$ (from case 1) $ => d' \neq w'^R$\\
        Hence, $s' \notin A$\\
    \end{enumerate}
\end{enumerate}

Hence A is not CFL.


\pagebreak


\section{Question 6}
\textbf{Prove the following stronger version of pumping lemma for CFLs: 
If $A$ is a CFL, then there is a number $k$ where if $s$ is any string in $A$ of length at least $k$ then $s$ may be divided into five pieces $s = uvxyz$, satisfying the conditions:
\begin{enumerate}
    \item for each $i\geq 0$, $uv^ixy^iz \in A$
    \item $v \neq \varepsilon$, and $y \neq \varepsilon$, and
    \item $|vxy| \leq k$.
\end{enumerate}}



\pagebreak


\section{Question 7}
\textbf{Give an example of a language that is not a CFL but nevertheless acts like a CFL in the pumping lemma for CFL (Recall we saw such an example in class while studying pumping lemma for regular languages). }

Consider the following languages:-\\
\begin{enumerate}
    \item $L_1 = ab^nc^nd^n$\\
        \textbf{Claim: }$L_1$ is not CFL\\
        \textbf{Proof: }Suppose $L_1$ is CFL, consider the language $L' = L_1 \bigcap b^*c^*d^* = b^nc^nd^n$\\
        Then, L' would also be regular since it is intersection of a CFL and a regular language (proved in ques 2 that intersection
        of regular langauage and CFL is CFL). But it is proved in class that L' is not a CFL. 
        Hence by contradiction $L_1$ is not CFL. 
    \item $L_2 = a^{k_1}b^{k_2}c^{k_3}d^{k_4} : k_1 \neq 1$\\
        $L_2$ is CFL because it is union of 2 regular languages : $b^*c^*d^* \bigcup a^2a^*b^*c^*d^*$ that is regular and all regular languages are CFL.
    \item $L_3 = L_1 \bigcup L_2$ \\
        $L_3$ is not a CFL as $L_3 \bigcap ab^*c^*d^* = L_1$ that is not a CFL and if $L_3$ were CFL, it should have been CFL by closure of union on CFL.
\end{enumerate}


Now, lets try to apply pumping lemma on $L_3$.\\
Let $p = 2$\\
Consider $\forall s \in L_3$\\
There are only 2 choices, either s is in $L_1$ or $L_2$ (as both have no intersection). Lets consider both of the cases seperately.\\
\begin{enumerate}
    \item $s \in L_1$\\
    Consider the partition of s = uvxyz, where $u = v = x = \epsilon, y = a, z = b^nc^nd^n (n > 0 as |s| \geq 2)$ \\
    Now, $\forall i \geq 0 : s' = uv^ixy^iz = a^ib^nc^nd^n$\\
    If i = 1 then $s'=s \in L_1 => s' \in L_3$ otherwise $s' $ is of the form  $a^{k_1}b^{k_2}c^{k_3}d^{k_4} : k_1 \neq 1$ i.e. $s' \in L_2 => S' \in L_3$\\
    $=> \forall i \geq 0 s' \in L_3$\\
    Hence $L_3$  satisfies pumping lemma.
    \item $s \in L_2$ \\
    This case is simple as $L_2$ is CFL, it should satisfy pupming lemma. Hence we are done.
\end{enumerate}
Hence provided an NCFL that satisfies pumping lemma.
\pagebreak


\end{document}