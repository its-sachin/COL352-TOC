\documentclass{article}
\usepackage[utf8]{inputenc}
\usepackage{amsmath}
\usepackage{mathtools}
\usepackage{bm}
\usepackage{graphicx}
\title{COL352: Assignment 2}
\author{Sachin 2019CS10722 \\
        Saurabh Verma 2019CS50129\\
        Sriram Verma}
\date{March, 2022}

\begin{document}

\maketitle


\section{Question 1}
\textbf{We say that a context-free grammar G is self-referential if for some non-terminal symbol $X$ we have $X \to^* \alpha X \beta$, where $\alpha, \beta \neq \varepsilon$. Show that a CFG that is not self-referential is regular.}



\pagebreak


\section{Question 2}
\textbf{Prove that the class of context-free languages is closed under intersection with regular languages. That is, prove that if \boldsymbol{$L_1$} is a context-free language and \boldsymbol{$L_2$} is a regular language, then \boldsymbol{$L_1 \cap L_2$}
is a context-free language. Do this by starting with a DFA}\\
\newline
Let us suppose there is a CFL L and a regular langauage R. The pushdown automata that accepts L be P=$(S_1,\Sigma,\Gamma,\delta_1,s_1,F_1)$ and the DFA accepting R be D=$(S_2,\Sigma,\delta_2,s_2,F_2)$. Now we have to show that the language $L\cap R$ is CFL. To show this it is enough to provide a PDA that accepts it. So we will construct such a PDA to prove that  $L\cap R$ is CFL. \\
\textbf{To Prove:} $L\cap R$ is CFL. \\
\textbf{Proof:} We will the above hypothesis by construction. The main idea behind the construction of the PDA is that we will run both the original PDA P for L and the DFA D in parallel on the input string and will only accept when the we reach an accepting state both in P and D. The construction of the PDA is described below:\\
\textbf{Construction:} Let the PDA which accepts $L\cap R$ be M=$(S,\Sigma,\Gamma,\delta,(s_1,s_2),F)$. Here S is $S_1 * S_2$ and F is $F_1 *F_2$. The transition function $\delta$ is described as follow: \\
For all the transitions $((p_1,a,\alpha),(p_2,\beta)) \in \delta_1$  and $(q_1,a,q_2) \in \delta_2$ add the transition $(((p_1,q_1),a,\alpha),((p_2,q_2),\beta))$ in $\delta$.\\
Also, for all the transitions $((p_1,\epsilon,\alpha),(p_2,\beta)) \in \delta_1$  and $\forall q \in S_2$ add the transition $(((p_1,q),a,\alpha),((p_2,q),\beta))$ in $\delta$.\\
Here $p_1,p_2 \in S_1$  $q_1,q_2 \in S_2$  $a \in \Sigma$  $\alpha , \beta \in \Gamma$\\
The accepting condition is that the final state reached after reading the input must belong to F.\\
Now our claim is that PDA M exactly recognises every string that is in $L\cap R$. \\
\textbf{Claim:} The PDA M constructed above exactly recognises strings in $L\cap R$. \\
\textbf{Proof:} We will have to show two things first that every string in $L\cap R$ is accepted by M. Lets prove this. Choose any string w$\in L\cap R$. Then its run on the DFA D would be something like $s_2,q1......q_k$ where $q_k \in F_2$, also w would take the PDA M from start configuration($s_1$) to an accepting configuration($q_{k^{'}}$) in some steps. By the way we have constructed the PDA M the computions of M and D will happen in parallel. So (s1,s2) is the start configuration of the PDA. First state in the tuple denotes the state that would have been in the PDA P and second state denotes the state that would have been in the DFA D after reading input upto some point. So when the PDA gets to wun on w. It would take M from from (s1,s2) to $(q_k,q_{k^{'}})$. Now this is an accepting configuration in M by the way we defined F$(F_1 * F_2)$. So all the string  w$\in L\cap R$ are accepted by M.\\
Also we have to prove that all the strings say w that are accepted by M should also be present in $ L\cap R$. We will accept w if it takes PDA M from $(s_1,s_2)$ to $(q_1,q_2)$ , $q_1 \in F_1$ and $q_2 \in F_2$. Also we have showed that M is parallely running P and D where first state in the tuple means the state reached in P and second state means the state reached in D after reading the input uptill that point. So after reading  w if the M is in state $(q_1,q_2)$, it would mean that after reading w P would have been in $q_1$ and D would be in $q_2$. $q_1 \in F_1$, so $w \in L$ also $q_2 \in F_2$ so $w \in R$ which implies $w \in L \cap R$.\\
\newline
Thus we have successfully constructed a PDA M which accepts $L \cap M$. Thus CFL's are closed under intersection with regular languages. Hence proved. \\
\pagebreak


\section{Question 3}
\textbf{Given two languages $L, L'$, denote by $$L||L' := \{x_1y_1x_2y_2 \dots x_ny_n \mid x_1x_2 \dots x_n \in L, y_1y_2\dots y_n \in L'\}$$ Show that if \boldsymbol{$L$} is a CFL and $L'$ is regular, then \boldsymbol{$L || L'$} is a CFL by constructing a PDA for \boldsymbol{$L || L'$}. Is \boldsymbol{$L || L'$} a CFL if both \boldsymbol{$L$} and \boldsymbol{$L'$} are CFLs? Justify your answer.}\\
\newline 
For the first part we have to show that T=$L||L'$ is a CFL if L is CFL and $L'$ is regular. Since L is CFL we are given a PDA P=$(S_1,\Sigma,\Gamma,\delta_1,s_1,F_1)$ which recognizes it. Also we have a DFA D=$(S_2,\Sigma,\delta_2,s_2,F_2)$ which recognises $L'$. These are the things given to us.\\
\textbf{To Prove:} T is CFL.\\
\textbf{Proof Idea:} To show that T is CFL we would have to construct a PDA say R which accepts it. Now the main idea behind the working of this machine is that after reading alphabet of the input tape it would guess whether the input came from L or $L'$ and would transition accordingly. Since PDA's can be non deterministic this construction would be fine. The specific construction of R is given as follows.\\
\textbf{Construction:} R=$(S,\Sigma,\Gamma,\delta,s,F)$. Here S is  $S_1 * S_2$ , F= $F_1 * F_2$. The transition function $\delta$ is described as below\\
$\delta((p_1 ,q_1),a,\alpha)$ would contain $((p_2,q_1),\beta)$ if $\delta_1(p_1,a,\alpha)$ contained $(p_2,\beta)$.\\
$\delta((p_1 ,q_1),a,\alpha)$ would contain $((p_1,q_2),\alpha)$ if $\delta_2(q_1,a)$=$\{q_2\}$.\\
Here $p_1,p_2 \in S_1$  $q_1,q_2 \in S_2$  $a \in \Sigma$  $\alpha , \beta \in \Gamma$\\
The accepting condition is that the final state reached after reading the input must belong to F.\\
Now our claim is that PDA R exactly recognises every string that belongs to T. \\
\textbf{Claim:} R recognises the language T.\\
\textbf{Proof:}\\
\\
\newline
Now let us see the second part of the question. Is $L||L'$ a CFL if both L and $L'$ are
CFLs? We will show that $L|| L'$ is not a CFL if L and $L'$ are CFls. \\
\textbf{To Prove:} $L|| L'$ is not a CFL if L and $L'$ are CFLs. \\
\textbf{Proof:} Giving a counter example will show that its not the case. Let us take two context free language and then we will show that $L || L'$ is not a CFL. L=$\{a^nb^n \ | \ n \geq 0\}$ . CFG for L is G=$(\{S\},\{a,b\},\{S \rightarrow aSb \ | \ \epsilon\},S)$. Thus L is CFL. Let $ L'=\{a^nb^{3n} \ | \ n \geq 0\}$ CFG for $L'$ is $G' =(\{S\},\{0,1\},\{S \rightarrow 0S111 \ | \ \epsilon\},S)$. Thus $L'$ is CFL.\\
Let us take the length of the string 4k where $k \geq 0$ ( Thats the length of the string nedeed because $L'$ contains string of the multiple of 4 and for $L || L'$ length of string of L and $L'$ must be same). So string in L is of the form $a^{2k}b^{2k}$ and string in $L' $ is of the form $0^{k}1^{3k}$. Hence we see $L || L'$ is $(a0)^k (a1)^k (b1)^{2k}$. Now our proof boils down to showing that $(a0)^k (a1)^k (b1)^{2k}$ is not CFL.\\
\textbf{Claim:} T=$(a0)^k (a1)^k (b1)^{2k}$ is not CFL.\\
\textbf{Proof:} We will use pumping lemma for CFL's. Let us consider the langauge is CFL. . Let p be the pumping length. Consider z=$(a0)^p (a1)^p (b1)^{2p} \in T$. Since $|z| > p$, there are u, v, w, x, y such that z = uvwxy, $|vwx| \leq p$, $|vx| > 0$ and $ uv^{i}wx^{i}y \in L$ for all $i \geq 0$. Since $|vwx| \leq p$ vwx would either contain a0,a1,b1 if vwx lies entirely in one of the 3 partitions else it could contain either a01,ab1 if vwx spans two partitions. In all these cases vwx will never contain all the four non-terminal symbols. Hence for every split if we pump up there would be imbalance in ateast one of the symbols. Hence the given language is not a CFL. \\
\newline
Since we have given a counter example in which the $L||L'$ is not a CFL given L and $L'$ are CFL we have successfully proved the hypothesis. 


\pagebreak


\section{Question 4}
\textbf{For \boldsymbol{$A \subseteq \Sigma^*$}, define 
\boldsymbol{$cycle(A) = \{yx \mid xy \in A\}$}
For example if \boldsymbol{$A = \{aaabc\}$}, then 
\boldsymbol{$cycle(A) = \{aaabc, aabca, abcaa, bcaaa, caaab\}$}
Show that if \boldsymbol{$A$} is a CFL then so is \boldsymbol{$cycle(A)$}} \\
\\
Let us suppose that we do have a CFG M=(V,T,P,S) for the langauge A in chomsky normal form. We know for a fact that since A is CFL it will have a CSG. Now to prove that cycle(A) is also a CFL. So we will construct a CFG for cycle(A) that would show that cycle(A) is CFL. \\
\textbf{To Prove:} Cycle(A) is CFL.\\
\textbf{Proof Idea:} Let us consider any string w in language A of the form $x_1 x_2$. Lets look at the parse tree of w. If we turn the parse tree upside down from the leftmost non terminal leaf from where $x_2$ starts then we will get the parse tree for $x_2 x_1$ , which is exactly what we are after. So through the construction described below we try to achieve this affect.\\
\textbf{Construction:} Let us consider the new grammar $M^{'} = (V^{'},T,P^{'},S_0)$ to accept cycle(A). Now here, $ V^{'} = V \cup \{ Z^{'} for \ Z \in V \} \cup \{ S_0\}$\\ P is defined as follow: 
\begin{itemize}
    \item All the rules in P
    \item $S_0 \rightarrow S$
    \item $S^{'}$ $\rightarrow$ $\epsilon$
    \item if  P  contained $ Z \rightarrow a$ add $S_0 \rightarrow aZ^{'}$
    \item if P has $Z \rightarrow XY$ add $Y^{'} \rightarrow Z^{'}X$  and $X^{'} \rightarrow YX^{'}$
\end{itemize}
Now we have constructed the grammar $M^{'}$. Whats left is to show that L($M{'}$) is exactly cycle(A). \\
\textbf{Claim:} L($M^{'}$) is same as cycle(A). \\
\textbf{Proof:} 

\pagebreak

\section{Question 5}
\textbf{Let $$A = \{wtw^R\mid w,t, \in \{0,1\}^* \text{ \ and \ } |w| = |t|\}$$ Show that $A$ is not a CFL. }




\pagebreak


\section{Question 6}
\textbf{Prove the following stronger version of pumping lemma for CFLs: 
If $A$ is a CFL, then there is a number $k$ where if $s$ is any string in $A$ of length at least $k$ then $s$ may be divided into five pieces $s = uvxyz$, satisfying the conditions:
\begin{enumerate}
    \item for each $i\geq 0$, $uv^ixy^iz \in A$
    \item $v \neq \varepsilon$, and $y \neq \varepsilon$, and
    \item $|vxy| \leq k$.
\end{enumerate}}



\pagebreak


\section{Question 7}
\textbf{Give an example of a language that is not a CFL but nevertheless acts like a CFL in the pumping lemma for CFL (Recall we saw such an example in class while studying pumping lemma for regular languages). }



\pagebreak


\end{document}